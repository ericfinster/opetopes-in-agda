%
%  opetopes-in-agda.tex - Opetopes in Agda
%

\documentclass{article}

\usepackage{stmaryrd}
\usepackage{tikz-cd}
\usepackage{graphicx}
\usepackage{amsmath}

\graphicspath{ {./imgs/} }

\usepackage{ucs}
\usepackage[utf8x]{inputenc}
\usepackage[T1]{fontenc}

\usepackage{latex/agda}

\usepackage{catchfilebetweentags}
\newcommand{\snippet}[1]{\ExecuteMetaData[latex/opetopes.tex]{#1}}

\newcommand{\pbmark}{\ar[dr, phantom, "\ulcorner" very near start, shift right=1ex]}
\newcommand{\polyarrowup}[2]{\arrow{r}[name=#1,inner sep=5pt,below]{}[description, inner sep=0pt]{|}[inner sep=5pt]{#2}}
\newcommand{\polyarrowdn}[2]{\arrow{r}[name=#1,inner sep=5pt]{}[description, inner sep=0pt]{|}[inner sep=5pt, below]{#2}}

\DeclareUnicodeCharacter{9654}{$\blacktriangleright$}
\DeclareUnicodeCharacter{8669}{$\rightsquigarrow$}
\DeclareUnicodeCharacter{9424}{@}

\newcommand\textPsi{$\Psi$}
\newcommand\textXi{$\Xi$}
\newcommand\textxi{$\xi$}
\newcommand\textDelta{$\Delta$}
\newcommand\textGamma{$\Gamma$}
\newcommand\textsigma{$\sigma$}
\newcommand\textSigma{$\Sigma$}
\newcommand\textPi{$\Pi$}
\newcommand\textrho{$\rho$}
\newcommand\textphi{$\varphi$}
\newcommand\texttau{$\tau$}
\newcommand\texteta{$\eta$}
\newcommand\textalpha{$\alpha$}
\newcommand\textbeta{$\beta$}
\newcommand\textepsilon{$\varepsilon$}
\newcommand\textkappa{$\kappa$}
\newcommand\textOmega{$\Omega$}
\newcommand\textmho{$\mho$}
\newcommand\textgamma{$\gamma$}
\newcommand\textlambda{$\lambda$}
\newcommand\textpi{$\pi$}
\newcommand\textmu{$\mu$}
\newcommand\texttheta{$\theta$}
\newcommand\textdelta{$\delta$}

\newcommand{\Set}{\mathcal{S}et}
\newcommand{\Poly}{\mathrm{Poly}}
\newcommand{\ext}[1]{\llbracket #1 \rrbracket}

\title{Opetopes in Agda}
\author{Eric Finster}

\begin{document}

\maketitle

\section{Introduction}
\label{sec:introduction}

The \emph{opetopes} are a family of polytopes introduced by Baez and
Dolan in \cite{BD97} in order to give a definition of weak
$n$-dimensional category.  These shapes are intimately related to
polynomial functors, a common mathematical tool used in the semantics
of inductive types for functional programming languages.

These notes contain a sketch of the definitions translated into the
language of Martin-L\"{o}f type theory, as implemented by the
programming language Agda.

\section{Polynomials}
\label{sec:polynomials}

Polynomial functors are an extremely important tool in computer science
for describing the semantics of inductively defined datatypes.  An overview
of the theoretical aspects can be found in \cite{GK13}.  For a presentation
directed more to the computer scientist, see \cite{AGHMM15}.

When speaking semantically, we will employ the representation of
indexed polynomials described in \cite{GK13}.  That is, a polynomial
is a diagram of sets of the following shape:

\[
\begin{tikzcd}
  & E \ar[dl, "\hat\tau"'] \ar[r, "\hat\rho"] & B \ar[dr, "\hat\gamma"] \\
  I & & & J
\end{tikzcd}
\]

\noindent The elements of $B$ are will be called \emph{constructors}, the
elements of $E$ \emph{places} and the elements of the two sets $I$ and
$J$ \emph{indices}.  

When translating this data into the language of type theory, it is often
represent maps between types as dependent types.  Under this translation,
the definition in Agda becomes

\snippet{polynomials}

\noindent and we recover the sets $B$ and $E$ by applying $\Sigma$.
That is,

\snippet{eb}

\noindent and under this definition, the maps $\hat\gamma$ and $\hat\rho$ become
simply the first projections from these sigma types.

Hence, for each $i:I$, the type $\gamma \, i$ is the type of constructors
of output type $i$. In this definition, γ is the type of
\emph{constructors}, ρ is the type of \emph{places} and τ assigns a
\emph{type} to each place.

It is useful to develop a kind of graphical notation for depicting
the elements of a polynomial.  So let us suppose we have some
$j : J$.  An element $c : \gamma \, j$ is a constructor whose
\emph{output type} is our given $j$.  Applying out dependent type
$\rho$ to this element, we have a set, whose elements are depicted
$p_1, p_2$ and $p_3$ in the diagram below.  Finally, the function
$\tau$ assigns to each of these elements an \emph{input type}, which
is an element of $I$.

\begin{center}
  \includegraphics[scale=0.75]{corolla}    
\end{center}

As shown in the diagram, then, we can picture a constructor $c$ of
type $j$ as a \emph{corolla}, that is, as a tree with a single
internal node corresponding labelled by $c$ and whose output edge is
labeled by the element $j$ and inputs by the typing funtion $\tau$
applied to each of the places.

The concept of a decoration is important.  It can be used to define
a functor corresponding to each polynomial, its \emph{extension}, which
can be defined as follows:

\snippet{extension}

An extremely important operation on polynomials is that they \emph{compose}.
The definition is given as follows:

\snippet{polycomp}

\subsection{W-Types}
\label{sec:w-types}

As we have seen, each polynomial $P$ induces a functor
$\ext{P} : \Set^I \to \Set^J$.  When $I=J$, so that the $\ext{P}$ is
an endomorphism of the category $\Set^I$, this functor in fact has an
initial fixed point, the \emph{W-type} corresponding to $P$, which is
defined as follows.

\snippet{wtypes}

The elements of the type $W P$ are the \emph{$P$-trees}, as
illustrated by the following diagram.

\subsection{Cartesian Morphisms}

There are various notions of morphisms between polynomials
that are in use in the literature.  Here we will be interested
in the \emph{cartesian morphisms}.  These are given by diagrams
of the following shape:

\[
\begin{tikzcd}
  I \ar[d, "f"'] & E \ar[l] \ar[r] \ar[d] \pbmark & B \ar[r] \ar[d] & J \ar[d, "g"] \\
  K & F \ar[l] \ar[r] & C \ar[r] & L
\end{tikzcd}
\]

The designation cartesian comes from the requirement that the middle
square is a pullback, as indicated in the diagram.  Translated into
Agda, this definition becomes:

\snippet{cartmorph}

\noindent Notice that the cartesian condition is expressed by the
requirement that the map on places $\rho$-map is in fact an equivalence.

In order to understand the algebra of cartesian morphisms and the
definitions which follow, it is helpful to elucidate the structure
enjoyed by the collection of polynomial functors and their interaction
with maps between types.  The structure we are interested in is that
of a \emph{framed bicategory}, whose basic theory is outline in \cite{S07}.

The basic idea is that we can think of a polynomial $P : \Poly I J$ as
a kind of generalized map from $I$ to $J$.  Moreover, for polynomials
$P$ and $Q$, the composition $P \odot Q$ givens these generalized maps
the structure of a category.  Let us therefore denote a polynomial $P$
as a decorated arrow as follows

\[
\begin{tikzcd}
  I \arrow{r}[description, inner sep=0pt]{|}[inner sep=5pt]{P} & J 
\end{tikzcd}
\]

\noindent in order to distinguish it from an ordinary map of types
$f : I \to J$.

Notice that our cartesian morphisms use both of these two kinds of
maps between types.  But we can naturally think of them, then, as
2-cells filling a square as follows:

\[
\begin{tikzcd}
  I \polyarrowup{U}{P} \ar[d, "f"'] & J \ar[d, "g"] \\
  K \polyarrowdn{D}{Q} & L
  \arrow[Rightarrow, from=U, to=D]{}[inner sep=5pt]{\alpha}
\end{tikzcd}
\]

\noindent and our notation is meant to reflect this geometry.  Since
cartesian morphisms are naturally regarded as two cells, our geometric
intuition would suggest that there are two ways of composing them.
This is indeed the case.  The ``vertical composition'' takes the following
form:

\[
\begin{tikzcd}[row sep=large]
  I \arrow{r}[name=U,inner sep=7pt,below]{}[description, inner sep=0pt]{|}[inner sep=5pt]{P} \ar[d, "f"'] & J \ar[d, "g"] \\
  K \arrow{r}[name=D,inner sep=7pt]{}[name=A,inner sep=8pt,below]{}[description, inner sep=0pt]{|}[inner sep=5pt, below]{Q} \ar[d, "h"']
  & L \arrow[Rightarrow, from=U, to=D]{}[inner sep=5pt]{\alpha} \ar[d, "k"] \\
  M \arrow{r}[name=B,inner sep=6pt]{}[description, inner sep=0pt]{|}[inner sep=5pt, below]{R}
  & N \arrow[Rightarrow, from=A, to=B]{}[inner sep=5pt]{\beta}
\end{tikzcd} \hspace{1cm} \Longmapsto \hspace{1cm}
\begin{tikzcd}[column sep=large]
  I \polyarrowup{U}{P} \ar[d, "h \circ f"'] & J \ar[d, "k \circ g"] \\
  K \polyarrowdn{D}{Q} & L
  \arrow[Rightarrow, from=U, to=D]{}[inner sep=5pt]{\alpha \blacktriangleright \beta}
\end{tikzcd}
\]

\noindent and can be defined in Agda as

\snippet{vertcomp}

The horizontal composition, on the other hand, will use the
composition of polynomials defined above, and is represented
geometrically as follows:

\[
\begin{tikzcd}
  I \polyarrowup{U}{P} \ar[d, "f"'] & J \polyarrowup{X}{R} \ar[d, "g"] & K \ar[d, "h"] \\
  L \polyarrowdn{D}{Q} & M \polyarrowdn{Y}{S} & N 
  \arrow[Rightarrow, from=U, to=D]{}[inner sep=5pt]{\alpha}
  \arrow[Rightarrow, from=X, to=Y]{}[pos=1.2,inner sep=5pt]{\beta}
\end{tikzcd} \hspace{1cm} \Longmapsto \hspace{1cm}
\begin{tikzcd}[column sep=large]
  I \polyarrowup{U}{P \odot R} \ar[d, "f"'] & J \ar[d, "h"] \\
  K \polyarrowdn{D}{Q \odot S} & L
  \arrow[Rightarrow, from=U, to=D]{}[inner sep=5pt]{\alpha || \beta}
\end{tikzcd}
\]

\noindent However, since it's definition requires the proof of a certain
extra coherence, we elide it for now, contenting ourselves to record the
type so that the reader can check how it corresponds to the double category
diagram above:

\snippet{horzcomp}

Next, we will need to spell out what it means for two cartesian morphisms
to be equal.  This is given by the following definition.

\snippet{carteq}

Some examples of cartesian morphisms which will be useful in what follows
are related to the unit and associative laws for composition of polynomials.
For example, as follows

\snippet{polymisc}

\section{Polynomial Monads}
\label{sec:polynomial-monads}

With the calculus of cartesian morphisms described above, we can now
give the definition of a polynomial monad.

\snippet{polymnds}

\section{The Slice Construction}
\label{sec:slice-construction}

We now present the Baez-Dolan slice construction. 

\snippet{slice}

The main thing to notice is how the indexing uses the multiplication
of the original monad.

We can also define the places and typing function for the slice
as follows

\snippet{sliceextra}

\bibliographystyle{plain}
\bibliography{opetopes-in-agda}

\end{document}

%%% Local Variables:
%%% mode: latex
%%% TeX-master: t
%%% End:
